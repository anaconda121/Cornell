\documentclass{article}
\usepackage{graphicx} % Required for inserting images
\usepackage{amsmath,amssymb}
\usepackage[margin=0.75in]{geometry}

\newcommand{\Var}{\mathrm{Var}}
\newcommand{\Cov}{\mathrm{Cov}}
\newcommand{\Corr}{\mathrm{Corr}}
\DeclareMathOperator{\EX}{\mathbb{E}}

\renewcommand{\baselinestretch}{1.5}

\title{\textbf{CS 2800 HW 12}}
\author{\textbf{Tanish Tyagi}}
\date{\textbf{December 4 2023}}

\begin{document}

\maketitle

\begin{center}
Collaborators: Timothy Li, Eric Zhou, Anthony Song
\end{center}

\section{Problem 1}

\subsection{Problem 1a}

$B_i$ assumes a geometric distribution with $p = \frac{i}{8}$.

$$
\EX(B_i) = \frac{1}{p} = \frac{1}{\frac{i}{8}} = \frac{8}{i}
$$

$$
\Var(B_i) = \frac{1 - p}{p^2} = \frac{1 - (\frac{i}{8})^2}{(\frac{i}{8})^2} = \frac{64 - 8i}{i^2}
$$

\subsection{Problem 1b}

To envision this relationship, we can start with $B_8$, which represents the number of boxes that must be purchased before a ticket from one of the \textbf{8} particular colors is collected. 
$B_7$ would then represent the number of boxes that need to be bought before a ticket that is one of the remaining seven colors is found. If we continue with this analogy, we can see each iteration of the summation will yield us a ticket of one of the remaining colors that we are looking for, until we have eventually collected a ticket of every single color.

\subsection{Problem 1c}

Using the linearity of expectation, $\EX(B) = \sum_{i = 1}^{8} \EX(B_i) = \sum_{i = 1}^{8} \frac{8}{i} = \frac{761}{35}$.

\noindent Using the linearity of variance, we can similarly state $\Var(B) = \sum_{i = 1}^{8}\Var(B_i) = \sum_{i = 1}^{8}\frac{64 - 8i}{i^2} = \frac{838034}{11025}$.

\section{Problem 2}

\subsection{Problem 2a}

$$f_S(x) = \frac{50 - x}{\binom{50}{2}}$$

$$f_L(x) = \frac{x - 1}{\binom{50}{2}}$$

\subsection{Problem 2b}

$$\EX(S) = \sum_{x = 1}^{50} \frac{x \cdot (50 - x)}{\binom{50}{2}} = 17$$

$$\EX(L) = \sum_{x = 1}^{50} \frac{x \cdot (x - 1)}{\binom{50}{2}} = 34$$

\subsection{Problem 2c}

$$\EX(S^2) = \sum_{x = 1}^{50} \frac{x^2 \cdot (50 - x)}{\binom{50}{2}} = 425$$

$$\EX(L) = \sum_{x = 1}^{50} \frac{x^2 \cdot (x - 1)}{\binom{50}{2}} = 1292$$

$$\Var(S) = \EX(S^2) - (\EX(S))^2 = 425 - 189 - 136$$

$$\Var(L) = \EX(L^2) - (\EX(L))^2 = 1292 - 1156 = 136$$

\subsection{Problem 2d}

$$\EX(S, L) = \sum_{S \in R_S}^{} \sum_{L \in R_L}^{} SL \cdot f_{S, L}(S, L)) = \sum_{S = 1}^{49} \sum_{L = S + 1}^{50} \frac{SL}{\binom{50}{2}} = 646$$

$$\Cov(S, L) = \EX(SL) - \EX(S)\EX(L) = 646 - 17 \cdot 34 = 68$$

\subsection{Problem 2e}

$$\Corr(S, L) = \frac{\Cov(S, L)}{\sqrt{\Var(S)}\sqrt{\Var(L)}} = \frac{68}{\sqrt{136}\sqrt{136}} = \frac{1}{2}$$

The correlation is positive, which makes sense--as the smaller value gets larger, it ensures that the larger value is picked out of a smaller set of larger numbers, increasing the likelihood the larger value is closer to 50. This means that as the smaller value gets larger, so does the larger value.

\section{Problem 3}

\subsection{Problem 3a}

For events $A_1, ..., A_n$ to be mutually independent, we need to show

$$
\Pr \left(\bigcap_{i \in S} A_i \right) = \prod_{i \in S} \Pr(A_i)
$$

\noindent where $S$ is an arbitrary subset of $[n]$ with $|S| = k$. The sample space is $|\mathcal{P}([n])| = 2^n$. We know $Pr(A_i) = \frac{1}{2}$ as there are two choices that have equal likelihood: either $i$ is in the subset or not. Therefore, $\prod_{i \in S} \Pr(A_i) = \left(\frac{1}{2} \right)^k$.

$$
\Pr \left(\bigcap_{i \in S}^{k} A_i \right) = \Pr(S \in A) = \sum_{j = 0}^{n} \Pr(|A| = j) \cdot \Pr(S \subseteq A \text{ given that } |A| = j)
$$

$$
\Pr(|A| = j) = \frac{\binom{n}{j}}{2^n}
$$

$$
\Pr(S \subseteq A \text{ given that } |A| = j) = \frac{\Pr(S \subseteq A \cap |A| = j)}{\Pr(|A| = j)} = \frac{\binom{n - j}{m - j}}{\binom{n}{j}} 
$$

\noindent Note that the $2^n$ terms in the numerator and denominator will cancel out. 

\noindent We can put this together to get:

$$
\frac{1}{2^n} \sum_{j = 0}^{k} \binom{n - j}{m - j} = \frac{1}{2^n} \sum_{j = 0}^{n - k} \binom{n - j}{m - j} = \frac{2^{n - k}}{2^n} = \left(\frac{1}{2} \right)^k
$$

\noindent Since we have shown  $\Pr \left(\bigcap_{i \in S} A_i \right) = \prod_{i \in S} \Pr(A_i)$ for an arbitrary subset, $A_1, ..., A_n$ events are mutually independent.

\subsection{Problem 3b}

We can construct the following summation:

$$\sum_{k = 0}^{n} \binom{n}{k} \cdot (2)^{n - k}$$

\noindent $\binom{n}{k}$ represents the number of $k$ length subsets for $A$, and $(2)^{n - k}$ represents the number of subsets that can constructed for $B$, given that it contains all $k$ elements of $A$. To get $\Pr(A  \subseteq B)$, we can divide this summation by $2^n \cdot 2^n$, which is the total possible subsets for $A$ and $B$.

\noindent Doing this, we get a final expression of:

$$\frac{1}{2^n \cdot 2^n}\sum_{k = 0}^{n} \binom{n}{k} \cdot (2)^{n - k}$$

\noindent We can use the binomial theorem to express the summation as $3^n$, giving us a final expression of $\frac{3^n}{4^n} = \boxed{\left(\frac{3}{4} \right)^n}$

\subsection{Problem 3c}

$$\Pr(A \subseteq B | A \subseteq C) = \frac{\Pr(A \subseteq B \cap A \subseteq C)}{\Pr(A \subseteq C)}$$

\noindent To find $\Pr(A \subseteq B \cap A \subseteq C)$, we can apply a similar process to the one in part b, this time with three subsets. Again, we construct the a summation:

$$\frac{1}{2^n \cdot 2^n \cdot 2^n}\sum_{k = 0}^{n} \binom{n}{k} \cdot (2)^{n - k} \cdot (2)^{n - k}$$

\noindent $\binom{n}{k}$ represents the number of $k$ length subsets for $A$, and $(2)^{n - k}$ represents the number of subsets that can constructed for $B$ and $C$, given that it contains all $k$ elements of $A$. To get $\Pr(A  \subseteq B)$, we can divide this summation by $2^n \cdot 2^n \cdot 2^n$, which is the total possible subsets that can be created for $A$, $B$, and $C$. This equals $\left(\frac{5}{8} \right)^n$. \noindent $\Pr(A \subseteq C) = \Pr(A \subseteq B) = \left(\frac{3}{4} \right)^n$.

\noindent Putting this together, we obtain a final expression of $\boxed{\left(\frac{5}{6} \right)^n}$

\subsection{Problem 3d}

$X$ assumes a binomial distribution with $p = \frac{1}{2}$. We can think of this scenario as a series of repeated trials, where a successful event is defined as whether an element $i \in [n]$ is added to set $A$.

\noindent $\EX(X) = n \cdot p = \boxed{\frac{n}{2}}$

\noindent $\Var(X) = \EX(X^2) - (\EX(X))^2 = np(1 - p) = \frac{n}{2}\left(\frac{1}{2}\right) = \boxed{\frac{n}{4}}$

\subsection{Problem 3e}

$Y$ assumes a binomial distribution with $p = \frac{1}{4}$. We can think of this scenario as a series of repeated trials, where a successful event is defined as whether an element $i \in [n]$ is added to sets $A$ and $B$.

\end{document}
