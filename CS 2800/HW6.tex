\documentclass{article}
\usepackage{graphicx} % Required for inserting images
\usepackage{amsmath}
\usepackage{amssymb}
\usepackage{amsthm}
% \usepackage[margin=0.75in]{geometry}

\title{\textbf{CS 2800 HW 6}}
\author{Tanish Tyagi \\ Collaborators: Timothy Li, Anthony Song, Eric Zhou}
\date{October 11, 2023}

\begin{document}

\maketitle

\section{Problem 1}

\subsection{Problem 1a}

We know that $\mathcal{F}(\mathbb{N})$ is countable if we can construct an injection $f: \mathcal{F}(\mathbb{N}) \rightarrow Y$, where $Y$ is a countable set. From week 6 discussion, we proved that the set $S = \bigcup_{i \in \mathbb{N}} S_{i}$ is countable, $S_i$ are countable sets. We also know that the $|\mathbb{N}|$ is countable. We can apply this to say for $S = \bigcup_{i \in \mathbb{N}} \mathbb{N}^i$ is countable. \\

Now, if we can construct an injection from $f: \mathcal{F}(\mathbb{N}) \rightarrow Y$, where $Y =\bigcup_{i \in \mathbb{N}} \mathbb{N}^i$, then we know that $|\mathcal{F}(\mathbb{N})| \le |Y|$, meaning that $\mathcal{F}(\mathbb{N})$ is countable as long as $Y$ is countable. \\

The function we can construct is as follows: $f(A) = B$, where $A \in \mathcal{F}(\mathbb{N}) = \{a_1, a_2, a_3, ..., a_n\}$, an arbitrary finite subset of $\mathcal{F}(\mathbb{N})$, and $B = (a_1, a_2, a_3, ..., a_n)$, an arbitrary ordered list. The definition of injectivity states that $x_1 \neq x_2 \Rightarrow f(x_1) \neq f(x_2)$. If $x_1 \neq x_2$, then it is either possible that $|x_1| \neq |x_2|$ or $|x_1| = |x_2|$. In the second case, we know that at least some term has to differ to maintain our initial premise that $x_1 \neq x_2$. In the first case, if $|x_1| \neq |x_2|$, then we can conclude $f(x_1) \neq f(x_2)$ as $f(x_1)$ and $f(x_2)$ will have a different number of elements. In the second case, while $|f(x_1)| = |f(x_2)|$, there has to be some element $x \in x_1$, but not in $x_2$ or vice versa. Therefore, if $x$ is only in $x_1$, it will be an element of $f(x_1)$ but not $f(x_2)$, proving that $f(x_1) \neq f(x_2)$. We can use similar logic to conclude that if $x$ is only in $x_2$, it will be an element of $f(x_2)$ but not $f(x_1)$, again showing that $f(x_1) \neq f(x_2)$. Therefore, in all possible cases, we have shown that if $x_1 \neq x_2$, then $f(x_1) \neq f(x_2)$, meaning that our function $f$ is indeed injective. \\

Since we have found an injection $f: \mathcal{F}(\mathbb{N}) \rightarrow \bigcup_{i \in \mathbb{N}} \mathbb{N}^i$, we know that $|\mathcal{F}(\mathbb{N})| \leq |\bigcup_{i \in \mathbb{N}} \mathbb{N}^i|$. We know that $\bigcup_{i \in \mathbb{N}} \mathbb{N}^i$ is countable, meaning that we can conclude $\mathcal{F}(\mathbb{N})$ is also countable. $\blacksquare$

\subsection{Problem 1b}

The error with this proof lies in the definition of $D$, and that it is possible for $D$ not to be finite, which means that it would not even be in the codomain. So even if it could not be mapped to, $f$ could still be surjective. For example, if $f(k) = \emptyset$ for all $k \in \mathbb{N}$, then $D = \mathbb{N}$, as $k \notin f(k)$ for all $k \in \mathbb{N}$. Here, $D$ is not an element of the codmain, which means it does not matter if $D$ is in the range of $f$. This leads to a logical breakdown of the proof. It is important to note that disproving this proof does not show that $\mathcal{F}(\mathbb{N})$ is countable.

\section{Problem 2}

\subsection{Problem 2a}

\begin{center}

\includegraphics[scale = 0.6]{2800 hw 6 2a.png}

\end{center}

\subsection{Problem 2b}

To prove that $\preceq_{L}$ is a partial order on $S \times S$, we need to show that $\preceq_{L}$ is reflexive, anti-symmetric, and transitive on $S \times S$.

\begin{enumerate}

\item Reflexive: We need to show that $(s_1, t_1) \preceq_{L} (s_1, t_1)$ for any arbitrary $(s, t) \in S \times S$, meaning that $(s_1\preceq s_1) \land (s_1 \npreceq s_1 \lor t_1 \preceq t_1)$.

By the definition of $\preceq_{L}$ on $S \times S$, we need to show that $s_1 \preceq s_1$ is true, and that either $s_1 \npreceq s_1$ or $t_1 \preceq t_1$ is true. We know that $\preceq$ is a poset, which means that $s_1 \preceq s_1$ has to be true as $\preceq$ has to be reflexive by the definition of posets. Likewise $t_1 \preceq t_1$. Therefore, $(s_1\preceq s_1) \land (s_1 \npreceq s_1 \lor t_1 \preceq t_1)$ will evaluate to true, meaning that $(s_1, t_1) \preceq_{L} (s_1, t_1)$. Therefore $\preceq_{L}$ is reflexive.

\item Anti-symmetric: We need to show that for any $(s_1, t_1), (s_2, t_2) \in S$, if $(s_1, t_1) \preceq_{L} (s_2, t_2) \land (s_2, t_2) \preceq_{L} (s_1, t_1)$ then $(s_1, t_1) = (s_2, t_2)$.

Let's assume that $(s_1, t_1) \preceq_{L} (s_2, t_2) \land (s_2, t_2) \preceq_{L} (s_1, t_1)$ is true. This tells us that $s_1 \preceq s_2$ and $s_2 \preceq s_1$, and since $\preceq$ is a poset, it has to be anti-symmetric, meaning that $s_1 = s_2$ given that $s_1 \preceq s_2$ and $s_2 \preceq s_1$.

Since $s_1 = s_2$ and $\preceq$ is reflexive, then $s_2 \preceq s_1$, meaning that $s_1 \npreceq s_2$ is false. We can use similar logic to say that $s_2 \preceq s_1$, meaning that $s_2 \npreceq s_1$ is also false. Given our initial assumption, we know that $(s_1, t_1) \preceq_{L} (s_2, t_2) \land (s_2, t_2) \preceq_{L} (s_1, t_1)$. In order for this to be true, this means that $t_1 \preceq t_2$ and $t_2 \preceq t_1$ by the definition of $(S \times S, \preceq_{L})$. Since $\preceq$ has to be a poset, it has fulfill the property of anti-symmetry, meaning that $t_1 = t_2$ if $t_1 \preceq t_2$ and $t_2 \preceq t_1$.

Now that we have shown that $s_1 = s_2$ and $t_1 = t_2$, we can conclude that $(s_1, t_1) = (s_2, t_2)$. Therefore, $\preceq_{L}$ is anti-symmetric.

\item Transitive: We need to show that $(s_1, t_1) \preceq_{L} (s_2, t_2) \land (s_2, t_2) \preceq_{L} (s_3, t_3) \Rightarrow (s_1, t_1) \preceq_{L} (s_3, t_3)$.

Let's assume $(s_1, t_1) \preceq_{L} (s_2, t_2) \land (s_2, t_2) \preceq_{L} (s_3, t_3)$ is true. Then, we know the following to be true:

\vspace{-15pt}
\begin{align*}
1: (s_1 \preceq s_2) \land (s_2 \npreceq s_1 \lor t_1 \preceq t_2) \\
2: (s_2 \preceq s_3) \land (s_3 \npreceq s_2 \lor t_2 \preceq t_3) 
\end{align*}

The first parts of these statements tell us that $s_1 \preceq s_2$ and $s_2 \preceq s_3$. Since $\preceq$ is a poset, this means it must be transitive, meaning that $s_1 \preceq s_3$.

Now we need to show $(s_3 \npreceq s_1 \lor t_1 \preceq t_3)$. We can split this into two cases: either $s_3 \npreceq s_1$ or $s_3 \preceq s_1$.

\begin{enumerate}

\item $s_3 \npreceq s_1$

Since $s_3 \npreceq s_1$ is true, this means $(s_3 \npreceq s_1 \lor t_1 \preceq t_3)$ overall is true as well by the definition of logical or.

\item $s_3 \preceq s_1$

Previously we showed $s_1 \preceq s_3$, and given that $s_3 \preceq s_1$, by the anti-symmetric of partial orders, $s_1 = s_3$. We also previously showed that $s_1 \preceq s_2$ and $s_2 \preceq s_3$. We can combine this with $s_1 = s_3$ to say $s_1 = s_2 = s_3$, as $s_2$ is `compressed' within $s_1$ and $s_3$. 

By the above equality and reflexivity, $s_2 \preceq s_1$. This means that in order for the second part of statement 1 to be true, $t_1 \preceq t_2$. We can use similar logic to say $s_3 \preceq s_2$, meaning that in order for the second part of statement 1 to be true, $t_2 \preceq t_3$. Since $\preceq$ is a poset, it has to be transitive, meaning that if $t_1 = t_2$ and $t_2 = t_3$, then $t_1 = t_3$. Therefore, $(s_3 \npreceq s_1 \lor t_1 \preceq t_3)$ is also true.

\end{enumerate}

In all possible cases, we have now shown $(s_3 \npreceq s_1 \lor t_1 \preceq t_3)$  to be true. As we have previously shown $s_1 \preceq s_3$, we know that $(s_1 \preceq s_3) \land (s_3 \npreceq s_1 \lor t_1 \preceq t_3)$ is also always true. Therefore, $\preceq_{L}$ is transitive.

\end{enumerate}

As we have now proved $\preceq_{L}$ is reflexive, anti-symmetric, and transitive on $S \times S$, this means that $\preceq_{L}$ is a partial order on $S \times S$. $\blacksquare$

\end{document}
